\documentclass[../summaries.tex]{subfiles}

\begin{document}

\subsection{Citation}
Wilson, Susan G., and Ivan Abel. "So you want to get involved in e-commerce." Industrial marketing management 31.2 (2002): 85-94.

\subsection{Summary}
The advent of the Internet and search engines has fundamentally altered the advertising game. While many papers discuss advertising within the search engine (i.e. sponsored search), businesses can also advertise a brand on their company website. However, that website must stand out from search engine results in order to attract users. How can a business boost the rank of their website in major search engines? The answer may save or sink the entire business.

As discussed in the first paper on PageRank [Brin, Page], search engines crawl the web and rank sites based on multiple factors. Recall also that pages with a low rank (top of page) are significantly more likely to be viewed, and those beyond the top ten are virtually dead. Businesses can craft their web page to earn a high ranking in a few different ways:

\begin{enumerate}
\item Register your site with as many search engines as possible. While the crawlers try to find new pages, manually submitting the URL to the engine's index ensures a swift discovery.
\item Use keywords in the Title of your webpage and make it as descriptive as possible. This will increase the likelihood of matching a keyword query.
\item Use META tags to provide extra information about your page. These tags are invisible to the reader.
\item Place important information near the top of the page because search engines assume the top portion is indicative of the entire page.
\item Give reputable sites (i.e. product review sites, news sites) a compelling reason to link to your website. This link will tell the search engine that your website is high quality and thus boost your rank.
\end{enumerate}

If your business site can achieve a high rank in search results, then it will receive more visitors. But that is only half the battle. You must design a compelling website that delights the user and it must change dynamically over time. If your site is a simple static page, users have no reason to visit again. Some successful commerce sites show new products upon each visit or a new sale or review of a product. They also implement digital catalogs where users can browse products from the comfort of home. An interactive website that appears in the top search results will serve as an excellent advertisement for the business.

\end{document}
