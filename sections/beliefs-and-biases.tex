\documentclass[../summaries.tex]{subfiles}

\begin{document}

\subsection{Citation}
White, Ryen. "Beliefs and biases in web search." Proceedings of the 36th international ACM SIGIR conference on Research and development in information retrieval. ACM, 2013.

\subsection{Summary}
Bias is a common human trait that occurs when a person's beliefs affect their actions and decisions. It turns out that these biased actions and decisions can manifest themselves through information retrieval systems--it occurs when a user searches for or accepts information that strays significantly from the truth. This paper sheds light on both how humans influence information retrieval systems, and how information retrieval systems influence humans.

To study search-related biases, three approaches were taken: an exploratory retrospective survey, human labeling of captions on a search engine, and a large-scale log analysis of that same search engine.

Search engines themselves can be biased. The results of this study found that search engines were significantly more likely to present captions and results that answered a yes or no question positively. On top of this, positive results were usually ranked above negative results in a search query. This meant that users were significantly more likely to accept positive results than negative results. Across all observed clicks in the data set analyzed, 41.1\% were on a positive result, while only 16.3\% were on a negative result--which means that positive answers are almost three times as common!

The bias towards positive results was actually irrespective of the actual truth. The researchers obtained a list of answers to medical questions -- provided by verified professionals -- and cross-referenced that data with answers that were accepted by users on search engines. The results show that if users trusted the top ranked result provided by the search engine, they obtained the correct answer only 45\% of the time. Furthermore, even when users referred to multiple sources before accepting an answer, their opinions never changed. As a result, researchers concluded that the primary motivation for pursuing a secondary source of information was to simply validate the first source.

\end{document}
