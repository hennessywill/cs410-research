\documentclass[../summaries.tex]{subfiles}

\begin{document}

\subsection{Citation}
Evans, David S. "The online advertising industry: Economics, evolution, and privacy." Journal of Economic Perspectives (2009).

\subsection{Summary}
In 2015, media companies around the globe spent \$545 billion dollars on advertisements. Historically, these ads have found a home in newspapers and TV commercials, but this model is under attack by the new era of online advertisements. Although Internet ads were born in the 1994, they have grown at an astonishing rate; ZenithOptimedia expects the Internet to account for more than one-third of U.S. ad spending in 2017, representing 400\% growth since 2007. Indeed, many major newspaper businesses have gone out of business. But while this revolution has crippled some businesses, it has created new efficiencies and opportunities for the global economy by significantly reducing the transaction cost for merchants to find consumers.

Advertising is a matching game between merchants and consumers. The old model required merchants to buy a million newspaper ads with the hope that a small fraction of the population is interested. The search engine has changed this game entirely. Merchants are now able to identify individual consumers who are interested in the product by matching their search queries or registered account information. This ability to detect a consumer's interest and intent to purchase a product has transformed advertising campaigns from mass market tools into focused, personalized ads. The search engines are an intelligent intermediary in this lucrative matching game.

So how do these businesses play the game? Typically, a company will start with an objective (like "increase sales of product X" or "make our brand more friendly") and set a budget to achieve this goal. Advertisers will then divide this budget amongst the various forms of media:  online, television, radio, magazines, newspapers, etc. based primarily on the expected rate of return for each medium. In a competitive marketplace, modern businesses have no choice but to take advantage the cost-efficient advertising offered by the online intermediary search engines. These online ads further increase their rate of return by directly linking the consumer to the merchants online portal to purchase goods.

In summary, the Internet's share of ad spending has grown over the past decade and will continue to grow because search engines provide a more cost-efficient, targeted method of matching consumers with merchants. Modern businesses must adopt this technology or risk becoming obsolete.

\end{document}
