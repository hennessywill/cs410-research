\documentclass[../summaries.tex]{subfiles}

\begin{document}

\subsection{Citation}
Rutz, Oliver J., Michael Trusov, and Randolph E. Bucklin. "Modeling indirect effects of paid search advertising: which keywords lead to more future visits?." Marketing Science 30.4 (2011): 646-665.

\subsection{Summary}
When you advertise your business on search engines, the most profitable metric is conversion rate because that corresponds to revenue for your company. However, there is another more powerful metric:  direct type-in visitation. This metric measures the number of users that return to your website directly sometime after seeing your advertisement. This is extremely valuable to a business because it means their brand name stuck in the consumer's head and they later singled out your company for shopping. This is customer loyalty.

The authors attempt to quantify this indirect benefit of online advertising so that business owners can more accurately calculate the return on their advertising investment. The study begins with a dataset from an e-commerce website in the automotive industry. This dataset was chosen because the indirect effects will be especially pronounced over the month-long research period that a typical car-shopper endures. The researches employ a Bayesian elastic net to calculate indirect effects at a keyword-level for thousands of keywords. They also designed an original text-classifier based on their knowledge of the business domain and web design.

Of the 3,186 keywords, 599 were successfully linked to fluctuations in direct type-in traffic. The insignificant keywords include very specific searches like "Toyota Avalon specs". These significant keywords include the company's brand name (i.e. Toyota) and broad words like "buy car", all resulting in more direct type-in visits to the automotive website. But how much this worth to the automotive company? First, the additional visits to their website generates marginal ad revenue of approximately \$0.26 per direct type-in visitor. This yields \$90,385 which is half of the entire annual paid search budget. Quite valuable indeed. Still, this profit estimation is far too conservative (likely by tens of thousands of dollars) because it neglects the additional car sales that result from these direct type-in visits.

% interesting:  the above significant keywords state that brand-specific keywords are very valuable. You read a previous paper saying brand-specific keywords were valuable as well. Apparently this brand keyword value extends from direct effects into these indirect effects. Discuss in presentation.

This all goes to show that while search engine dashboard metrics like conversion rates and click-through rates are valuable, they are not holistic. Your online advertising efforts produce additional, indirect benefits when the consumer initiates direct type-in visits to your site long after seeing your search advertisement.

\end{document}
