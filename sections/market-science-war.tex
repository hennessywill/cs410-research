\documentclass[../main-paper.tex]{subfiles}

\begin{document}

\subsection{Citation}
Van Couvering, Elizabeth. "Is relevance relevant? Market, science, and war: Discourses of search engine quality." Journal of Computer-Mediated Communication 12.3 (2007): 866-887.

\subsection{Summary}
The author conducted a study to understand the motivating factors that drive search engine producers (SEP) to make changes to the engine. Between November 2002 and May 2004, the author interviewed eleven SEPs, including senior engineers and technical executives who direct future code development. These interviewees worked at every major and minor search engine on the market:  Google, Yahoo!, MSN, Ask Jeeves, AOL, and more. Each individual was interviewed over the phone for 1-2 hours in a semistructured, in-depth format. Questions probed for specific instances of change to the search engine and inquired for the motivation behind that change. The text transcript of each interview was categorized to identify themes, from which the author determined two major schemas that motivate the development of search engine technology:\\

\textbf{1) The Market Schema}

Throughout these interviews, the most common category of motivator was the market schema, which includes revenues, costs, competition, and other business issues. In explaining this motivation, interviewees regarded their search engine as a commercial service competing for users in the marketplace. Thus, the primary motivator for SEPs is financial profit and that metric is linked to search engine quality via its direct correlation with customer satisfaction. Many of the technical changes were developed to increase profit.\\

\textbf{2) The Science/Technology Schema}

The second most common category of motivator was the science schema, which includes experimentation, measurement, feasibility, and objectivity. This motivator defined quality as relevance, or the ability to answer a user's question, and was defined by data-driven metrics. Many of the changes were developed to improve search result relevance, recall, or precision.\\

These interviews also revealed a subjective component of the search engines: blacklists, whitelists, and topic-specific weights. This censoring was often dictated by executives to respond to current events, but it is arbitrary and non-scientific. Still, these practices are accepted by the SEP because they strive to boost relevance.

This paper impacts all modern businesses because it describes the convoluted environment in which they compete for clicks. For many companies, the search engine is the primary portal through which consumers are reached. However, this paper shows that the portal is controlled by employees who seek to maximize profit or relevance for the search engine company, not for the online businesses who depend on the search engine. Thus, an online business must actively monitor and effect their position in the search engine rankings to maintain a steady flow of customers.

\end{document}
